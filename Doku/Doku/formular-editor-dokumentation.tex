\documentclass[a4paper,11pt]{article}


\usepackage[ngerman]{babel}
\usepackage[utf8]{inputenc}
\usepackage[T1]{fontenc}
\usepackage{geometry}
\usepackage{helvet}
\usepackage{graphicx}
\usepackage{enumitem}
\usepackage{setspace}
\usepackage{booktabs}
\usepackage{longtable}
\usepackage{multirow}
\usepackage{fancyhdr}
\usepackage{bookmark} 
\usepackage{hyperref} 


\hypersetup{
    bookmarks=true,        
    bookmarksnumbered=true, 
    colorlinks=false,       
    hidelinks=true,        
    pdfborder={0 0 0},      
    pdftitle={Dokumentation zur betrieblichen Projektarbeit},
    pdfauthor={Mustafa Shahin},
    pdfsubject={Formular-Editor Schnittstelle},
    pdfkeywords={Angular, Formular-Editor, GSCS},
    pdfstartview={FitH},    
    pdfdisplaydoctitle=true 
}


\renewcommand{\familydefault}{\sfdefault}

\geometry{
  top=3.3cm,
  left=2.5cm,
  bottom=2.5cm,
  right=2.5cm,
  headheight=1.8cm,
  footskip=1.3cm
}


\singlespacing


\begin{document}


\pagestyle{fancy}
\fancyhf{}
\fancyhead[R]{\includegraphics[height=1cm]{green_solutions_logo}}
\fancyhead[L]{Formular-Editor Schnittstelle}
\fancyfoot[C]{\thepage}
\renewcommand{\headrulewidth}{0.4pt}
\renewcommand{\footrulewidth}{0.4pt}


\begin{titlepage}
\thispagestyle{empty}

\begin{center}
\includegraphics[width=0.8\textwidth]{ihk_logo}

\vspace{0.5cm}
{\large Abschlussprüfung Sommer 2025}\\
\vspace{0.5cm}
{\large Fachinformatiker für Anwendungsentwicklung}\\
\vspace{1cm}
{\Large\textbf{Dokumentation zur betrieblichen Projektarbeit}}\\
\vspace{1cm}
{\Huge\textbf{Formular-Editor Schnittstelle}}\\
\vspace{0.5cm}
{\Large\textbf{Implementierung eines dynamischen Formular-Editors}}\\
\vspace{0.1cm}
{\Large\textbf{für die GSCS der Green Solutions Software GmbH}}\\
\end{center}

\vspace{0.3cm}
\begin{center}
{\large Abgabedatum: Oldenburg, den 10.05.2025}
\end{center}

\vspace{0.3cm}
\begin{center}
\begin{tabular}{l}
{\large\textbf{Prüfungsbewerber:}}\\
{\large Mustafa Shahin}\\
{\large Grenadierweg 2}\\
{\large 26129 Oldenburg}\\
\\
{\large\textbf{Ausbildungsbetrieb:}}\\
{\large Green Solutions Software GmbH}\\
{\large Eva-Lessing-Straße 6}\\
{\large 26160 Bad Zwischenahn}
\end{tabular}
\end{center}


\begin{center}
\includegraphics[width=0.8\textwidth]{green_solutions_logo}
\end{center}
\end{titlepage}


\tableofcontents
\newpage

\section{Einleitung}

\subsection{Projektumfeld}

Diese Projektdokumentation beschreibt den Ablauf meines IHK-Projekts im Rahmen meiner Ausbildung zum Fachinformatiker für Anwendungsentwicklung bei der Green Solutions Software GmbH (GSS-GmbH).

Die GSS-GmbH ist ein innovatives Softwareunternehmen mit 50 Mitarbeitern, das Software primär für die grüne Branche anbietet. Die zentrale Software der GSS-GmbH ist die green solutions cloud software (GSCS), welche relevante Kanäle im Online- und Offline-Bereich in einem System zusammenfügt. Von Webshop, Website, Newsletter, Dropshipping, CRM, SEO, SEA und Social Media bis hin zu Beschilderungen, Apps und digitaler Kundenbindung integriert die GSCS diverse Funktionalitäten.

Im Zuge der Modernisierung wurde das Frontend der GSCS von einer MVC-Architektur mit jQuery auf Angular umgestellt. Dadurch ist die bisherige Formularverwaltung nicht mehr kompatibel mit der neuen Architektur. Formulare sind ein zentrales Element der GSCS und werden in verschiedenen Bereichen eingesetzt, von Kundendaten bis hin zu Produktkonfigurationen.

\subsection{Projektziel}

Ziel des Projekts ist es, einen dynamischen Formular-Editor für die GSCS zu entwickeln und zu implementieren. Der Editor soll es Benutzern mit speziellen Rechten (Designer-Rechten) ermöglichen, Formulare innerhalb des Systems flexibel zu gestalten und anzupassen. Die Umsetzung erfolgt mit dem .NET Framework im Backend und Angular im Frontend.\\
Der Editor wird als Floating Window implementiert und soll folgende Kernfunktionalitäten bieten:

\begin{itemize}
\item Hinzufügen und Entfernen von Formularfeldern
\item Anpassung der Feldbreite und -größe
\item Definition von Pflichtfeldern
\item Festlegung von Validierungsregeln
\item Speichern von Formular-Templates
\end{itemize}\\
Die Änderungen sollen in Echtzeit visualisiert und nach dem Speichern sofort im System verfügbar sein.

\subsection{Projektbegründung}

Die Umstellung des Frontends der GSCS auf Angular erfordert eine Neuentwicklung der Formularbearbeitung. Der bisherige Formular-Editor ist nicht mehr kompatibel mit der neuen Angular-Architektur. Formulare sind jedoch ein zentraler Bestandteil der GSCS und werden in verschiedenen Bereichen wie Kundenverwaltung, Produktkonfiguration und Bestellabwicklung eingesetzt.
Eine flexible und benutzerfreundliche Möglichkeit zur Anpassung von Formularen ist für die Kunden der GSS-GmbH von großer Bedeutung, da verschiedene Branchen unterschiedliche Anforderungen an Formulare haben. Die Entwicklung eines neuen, auf Angular basierenden Formular-Editors ist daher unerlässlich, um die Funktionalität der GSCS aufrechtzuerhalten und zu verbessern.

\subsection{Projektschnittstellen}

Die GSCS der GSS-GmbH basiert auf C\# mit ASP.NET und dem Entity Framework und wird auf einem IIS-Webserver mit einer SQL-Datenbank betrieben. Das Frontend wurde von MVC mit jQuery auf Angular umgestellt.\\
Die Benutzer des Formular-Editors sind primär Administratoren und Benutzer mit Designer-Rechten, die für die Anpassung und Erstellung von Formularen innerhalb des Systems verantwortlich sind.\\
Ein Code Review wurde in Vorbereitung der Abnahme mit einem weiteren Fachinformatiker für Anwendungsentwicklung der GSS-GmbH durchgeführt.\\
Genehmigt wurde das Projekt von der Geschäftsführung der GSS-GmbH. In regelmäßigen Meetings wurde der Abteilungsleiter der Entwicklung über den aktuellen Entwicklungsstand und den Fortschritt des Projekts informiert.

\subsection{Projektabgrenzung}

Der Formular-Editor wird in das bestehende GSCS-System integriert. Die Implementierung umfasst die Angular-Frontend-Umsetzung und die Integration in das bestehende Backend-System.\\
Folgende Einschränkungen wurden berücksichtigt:

\begin{itemize}
\item Die Bearbeitung von Formularen ist nur für Benutzer mit Designer-Rechten möglich
\item Bestimmte System-Kern-Formulare können nicht bearbeitet werden
\item Die Formularänderungen müssen mit der bestehenden Datenstruktur kompatibel sein
\end{itemize}

\section{Planung}

\subsection{Projektphasen}

Für das vorliegende Projekt standen insgesamt 80 Stunden für die Umsetzung zur Verfügung. Im Rahmen des Projektantrags wurde bereits eine strukturierte Aufteilung des Projekts in verschiedene Phasen durchgeführt:

\begin{enumerate}
\item Planung/Analyse (10h)
\item Entwurf (14h)
\item Implementierung (40h)
\item Testen und Abnahme (8h)
\item Dokumentation (8h)
\end{enumerate}

Die detaillierte Zeitplanung befindet sich im Anhang A1 Detaillierte Zeitplanung.

Die Umsetzung des Projekts erfolgte während der regulären Arbeitszeit. Die Durchführung des Projekts verlief parallel zum Berufsalltag.

\subsection{Ressourcenplanung}

Im Rahmen der Ressourcenplanung wurden für das Projekt alle notwendigen Faktoren wie Personal, Hard- und Software berücksichtigt. Bei der Wahl der eingesetzten Software wurde darauf geachtet, dass diese bereits Bestandteil der erworbenen Lizenzen der GSS-GmbH oder kostenlos erhältlich waren, um zusätzliche Kosten so gering wie möglich zu halten.

Die Entwicklung fand auf zwei PCs statt, einem Office-Rechner mit Windows 11 und 6 Bildschirmen und einem Homeoffice-Rechner mit Windows 11 und 3 Bildschirmen. Bei beiden wurde die Entwicklungsumgebung Visual Studio Professional 2022 verwendet. Für die Angular-Entwicklung kamen außerdem VS Code, Node.js und npm zum Einsatz. Zur Versionsverwaltung des Projekts wurden die bestehenden Git-Repositories der GSS-GmbH verwendet.

\subsection{Entwicklungsprozess}

Die Umsetzung des Projekts verlief nach dem Wasserfallmodell. Die einzelnen Phasen wurden nacheinander abgearbeitet, wobei es in einigen Phasen zu Überlappungen kam, um den Entwicklungsprozess zu optimieren.

\section{Analyse}

\subsection{Durchführung Ist-Analyse}

Das bisherige System nutzte Dotnet als MVC-Architektur, wobei die Formularbearbeitung mit Hilfe von jQuery umgesetzt wurde. Die Formulare wurden in der Datenbank als JSON-Struktur gespeichert und bei Bedarf dynamisch gerendert.

Bestehende Probleme im alten System:

\begin{itemize}
\item Keine Vorschau-Funktionalität während der Bearbeitung
\item Fehlende Drag-and-Drop-Funktionalität für einfachere Bedienung
\item Keine Möglichkeit zur Speicherung von Formular-Templates
\end{itemize}

\subsection{Durchführung Soll-Analyse}

Das Ziel des Projekts ist es, einen modernen, auf Angular basierenden Formular-Editor zu entwickeln, der die bestehenden Probleme behebt und neue Funktionalitäten bietet.

Folgende Anforderungen wurden definiert:

\begin{itemize}
\item Entwicklung eines modernen, benutzerfreundlichen Editors
\item Implementierung einer Echtzeit-Vorschau
\item Integration in das bestehende Angular-Frontend
\item Kompatibilität mit der bestehenden Datenstruktur
\item Drag-and-Drop-Funktionalität für einfache Bedienung
\item Speichermöglichkeit für Formular-Templates
\end{itemize}

\subsection{Durchführung der Wirtschaftlichkeitsanalyse und Amortisationsrechnung}

Durch die Modernisierung des Formular-Editors wird die Benutzerfreundlichkeit der GSCS erheblich verbessert. Dies führt zu einer höheren Kundenzufriedenheit.

Da der Formular-Editor ein integraler Bestandteil der GSCS ist und die Umstellung auf Angular bereits beschlossen war, war eine Neuentwicklung unumgänglich. Alternativen wie die Beibehaltung des alten Systems oder der Zukauf einer externen Lösung wurden aufgrund von Kompatibilitätsproblemen und hohen Lizenzkosten verworfen.

Die Kosten für die Entwicklung des Formular-Editors belaufen sich auf ca. 4.000€ (80 Stunden à 50€). Die erwartete Amortisationsdauer beträgt etwa 6 Monate durch gesteigerte Effizienz bei der Formularbearbeitung und reduzierte Supportanfragen.

\section{Entwurf}

\subsection{Entwurf der Benutzeroberfläche}

Die Benutzeroberfläche des Formular-Editors wurde als Floating Window konzipiert, das über dem eigentlichen Formular angezeigt wird. Das Design folgt dem Corporate Design der GSCS und integriert sich nahtlos in die bestehende Benutzeroberfläche.

Die Benutzeroberfläche wurde in folgende Bereiche unterteilt:

\begin{itemize}
\item Toolbox mit verfügbaren Formularfeldern (Textfeld, Zahlenfeld, Dropdown, Checkbox, Radio-Button, etc.)
\item Konfigurationsbereich für das ausgewählte Formularfeld
\item Vorschaubereich für das gesamte Formular
\item Aktionsbereich mit Buttons zum Speichern, Abbrechen und Zurücksetzen
\end{itemize}

\subsection{Entwurf der Komponentenstruktur}

Für die Implementierung des Formular-Editors wurde eine modulare Komponentenstruktur gewählt, die sich gut in die bestehende Angular-Architektur integriert. Folgende Hauptkomponenten wurden definiert:

\begin{itemize}
\item FormEditorComponent: Hauptkomponente, die das Floating Window und die Integration der anderen Komponenten handhabt
\item ToolboxComponent: Verwaltet die verfügbaren Formularfelder
\item FieldConfigComponent: Stellt die Konfigurationsmöglichkeiten für das ausgewählte Formularfeld bereit
\item FormPreviewComponent: Zeigt eine Echtzeit-Vorschau des Formulars an
\end{itemize}

\subsection{Entwurf der Datenhaltung}

Die Formulardaten werden in einer JSON-Struktur gespeichert, die sowohl die Feldtypen als auch deren Konfiguration und Validierungsregeln enthält. Diese Struktur ist kompatibel mit dem bestehenden Datenmodell und kann problemlos in der SQL-Datenbank der GSCS gespeichert werden.

\begin{verbatim}
{
  "formId": "customer-form",
  "formName": "Kundenformular",
  "fields": [
    {
      "id": "name",
      "type": "text",
      "label": "Name",
      "required": true,
      "width": 12
    },
    {
      "id": "email",
      "type": "email",
      "label": "E-Mail",
      "required": true,
      "width": 6
    }
  ]
}
\end{verbatim}

\section{Implementierung}

\subsection{Implementierung der Angular-Komponenten}

Die Implementierung der Angular-Komponenten erfolgte gemäß dem Entwurf. Besonderer Wert wurde auf Wiederverwendbarkeit und Testbarkeit gelegt. Die Komponenten wurden als eigenständige Module implementiert, die unabhängig voneinander getestet werden können.

Die FormEditorComponent wurde als Modal-Dialog implementiert, der über dem eigentlichen Formular angezeigt wird. Die Kommunikation zwischen den Komponenten erfolgt über Services und EventEmitter.

\subsection{Implementierung der Drag-and-Drop-Funktionalität}

Für die Drag-and-Drop-Funktionalität wurde die Angular CDK (Component Development Kit) verwendet. Diese ermöglicht es, Formularfelder einfach per Drag-and-Drop aus der Toolbox in das Formular zu ziehen und innerhalb des Formulars zu verschieben.

\subsection{Integration in das bestehende System}

Die Integration in das bestehende System erfolgte über einen eigenen API-Endpunkt, der die Speicherung und den Abruf von Formulardaten ermöglicht. Die Berechtigungsprüfung erfolgt über das bestehende Rechtesystem der GSCS.

\subsection{Implementierung der Konfiguration}

Die Konfiguration des Formular-Editors erfolgt über eine eigene Konfigurationskomponente, die es ermöglicht, globale Einstellungen wie verfügbare Feldtypen, Standard-Validierungsregeln, etc. festzulegen.

\section{Abnahme}

\subsection{Durchführung der Tests}

Zur Sicherstellung der Qualität wurden umfangreiche Tests durchgeführt:

\begin{itemize}
\item Unit-Tests für die einzelnen Komponenten und Services
\item Integrationstests für das Zusammenspiel der Komponenten
\item End-to-End-Tests für die Gesamtfunktionalität des Formular-Editors
\end{itemize}

Dabei wurden verschiedene Testszenarien durchgespielt, wie das Anlegen neuer Formulare, die Bearbeitung bestehender Formulare, die Validierung von Eingaben, etc.

\subsection{Code Review}

Vor der Abnahme wurde ein Code Review mit einem weiteren Entwickler der GSS-GmbH durchgeführt. Dabei wurden einige Verbesserungsvorschläge gemacht, die vor der finalen Abnahme umgesetzt wurden.

\subsection{Abnahme durch den Projektverantwortlichen}

Die Abnahme des Projekts erfolgte durch den Projektverantwortlichen Herrn Alexander Kelm. Dabei wurden alle Anforderungen geprüft und für erfüllt befunden. Der Formular-Editor wurde anschließend in die Produktivumgebung übernommen.

\section{Interne Dokumentationen}

\subsection{Erstellung der Entwicklerdokumentation}

Die Entwicklerdokumentation umfasst eine detaillierte Beschreibung der Architektur, der Komponenten und der Schnittstellen des Formular-Editors. Sie dient als Referenz für zukünftige Weiterentwicklungen und Wartungsarbeiten.

Die Dokumentation wurde in Markdown erstellt und im Git-Repository der GSS-GmbH abgelegt. Zusätzlich wurden die Komponenten und Services im Code ausführlich dokumentiert.

\subsection{Erstellung des Benutzerhandbuchs}

Das Benutzerhandbuch beschreibt die Bedienung des Formular-Editors aus Sicht der Endbenutzer. Es enthält detaillierte Anleitungen mit Screenshots zur Erstellung und Bearbeitung von Formularen, zur Konfiguration von Formularfeldern und zur Festlegung von Validierungsregeln.

Das Benutzerhandbuch wurde im Wiki-System der GSS-GmbH veröffentlicht und steht allen Mitarbeitern und Kunden zur Verfügung.

\section{Fazit}

\subsection{Soll-/Ist-Vergleich}

Die Umsetzung des Formular-Editors wurde erfolgreich abgeschlossen und alle definierten Anforderungen wurden erfüllt. Die Zeitplanung wurde eingehalten, lediglich in der Implementierungsphase gab es leichte Verzögerungen durch unvorhergesehene Schwierigkeiten bei der Integration in das bestehende System, die jedoch durch effizientere Arbeit in den anderen Phasen ausgeglichen werden konnten.

In der Testphase wurden einige kleinere Fehler entdeckt, die vor der finalen Abnahme behoben wurden. Insgesamt verlief das Projekt nach Plan und die Qualität des Endergebnisses entspricht den Erwartungen.

\subsection{Ausblick}

Der Formular-Editor wird in zukünftigen Versionen der GSCS weiterentwickelt. Geplante Erweiterungen umfassen:

\begin{itemize}
\item Erweiterung um zusätzliche Feldtypen (Datei-Upload, Datum/Uhrzeit, etc.)
\item Verbesserung der Vorschau-Funktionalität
\item Integration von Formular-Templates für verschiedene Branchen
\item Implementierung einer Versionierung für Formulare
\end{itemize}

\subsection{Gelerntes}

Während der Projektumsetzung konnte ich viele Erfahrungen im Bereich der Angular-Entwicklung sammeln. Die Arbeit mit Angular und TypeScript hat mir gezeigt, wie moderne Web-Frameworks die Entwicklung komplexer Anwendungen erleichtern können.

Die Herausforderungen bei der Integration in das bestehende System haben mein Verständnis für die Komplexität großer Softwareprojekte vertieft. Die enge Zusammenarbeit mit dem Projektverantwortlichen und anderen Entwicklern hat mir wertvolle Einblicke in die Projektorganisation und -kommunikation gegeben.

\newpage
\appendix
\section{Anhang}

\subsection{A1 Detaillierte Zeitplanung}

\begin{longtable}{lll}
\toprule
\textbf{Phase} & \textbf{Aktivität} & \textbf{Zeit (h)} \\
\midrule
\multirow{4}{*}{Planung/Analyse} & Durchführung Ist-Analyse & 2 \\
 & Durchführung Soll-Analyse & 5 \\
 & Analyse bestehender Formular Strukturen & 1 \\
 & Erstellung Lastenheft & 2 \\
\midrule
\multirow{4}{*}{Entwurf} & Entwurf der Benutzeroberfläche & 6 \\
 & Planung der Komponentenstruktur & 4 \\
 & Konzeption der State Management Strategie & 2 \\
 & Erstellung Pflichtenheft & 2 \\
\midrule
\multirow{4}{*}{Implementierung} & Modaler Editor & 12 \\
 & Formular-Konfiguration & 12 \\
 & System-Integration & 8 \\
 & Validierungslogik & 8 \\
\midrule
\multirow{4}{*}{Testen und Abnahme} & Entwicklung Testszenarien & 2 \\
 & Durchführung Tests & 2 \\
 & Code-Review & 2 \\
 & Abnahme & 2 \\
\midrule
\multirow{2}{*}{Dokumentation} & Technische Dokumentation & 4 \\
 & Benutzerhandbuch & 4 \\
\midrule
 & \textbf{Gesamt} & \textbf{80} \\
\bottomrule
\end{longtable}

\end{document}